\documentclass{article}

\usepackage{xeCJK}
\setCJKmainfont{SimSun}

\title{Personal Information Manager System API Document}
\author{李约瀚 \\ 14130140331 \\ qinka@live.com \\ qinka@qinka.pw}

\usepackage{listings}
\usepackage{hyperref}

% For Haskell
\lstnewenvironment{haskell}[1][]
{\lstset{ frame=tblr
        , breaklines           
        , basicstyle=\small\ttfamily  
        , flexiblecolumns=false  
        , basewidth={0.5em,0.45em}   
        , literate={+}{{$+$}}1 % plus  
            {/}{{$/$}}1                  
            {*}{{$*$}}1                  
            {=}{{$=$}}1                  
            {>}{{$>$}}1                  
            {<}{{$<$}}1                  
            {\\}{{$\lambda$}}1           
            {\\\\}{{\char`\\\char`\\}}1  
            {->}{{$\rightarrow$}}2       
            {>=}{{$\geq$}}2              
            {<-}{{$\leftarrow$}}2        
            {<=}{{$\leq$}}2              
            {=>}{{$\Rightarrow$}}2       
            {\ .}{{$\circ$}}2            
            {\ .\ }{{$\circ$}}2          
            {>>}{{>>}}2                  
            {>>=}{{>>=}}2                
            {|}{{$\mid$}}1               
            {/=}{{$\neq$}}1              
        , numbers=left%
        , #1%
    }%
    \csname lst@SetFirstLabel\endcsname}
{\csname lst@SaveFirstLabel\endcsname}
% For Java
\lstnewenvironment{java}[1][]
{\lstset{frame=tblr
        , breaklines           
        , basicstyle=\small\ttfamily  
        , flexiblecolumns=false  
        , basewidth={0.5em,0.45em} 
        , literate={+}{{$+$}}1 % plus  
                   {/}{{$/$}}1                  
                   {*}{{$*$}}1                  
                   {=}{{$=$}}1                  
                   {>}{{$>$}}1                    
                   {<}{{$<$}}1       
                   {>=}{{$\geq$}}2        
                   {<=}{{$\leq$}}2             
        , numbers=left%
        , #1
        }
    \csname lst@SetFirstLabel\endcsname}
{\csname lst@SaveFirstLabel\endcsname}
% For JSON
\lstnewenvironment{json}[1][]
{\lstset{ frame=tblr
        , breaklines           
        , basicstyle=\small\ttfamily  
        , flexiblecolumns=false  
        , basewidth={0.5em,0.45em}             
        , numbers=left%
        , #1%
        }
    \csname lst@SetFirstLabel\endcsname}
{\csname lst@SaveFirstLabel\endcsname}


\begin{document}
    \maketitle
    \newpage
    \tableofcontents
    \newpage
    
    % % Main
    \section{Application Text Data Interface}
    This section includes the application text data interfaces' definition.
    At the same time, this section will include the representation application datas' typeclass/interface.
    
    For the common texture data interface, JSON will be used when datas are needed to be represent to text.
    
    \subsection{[Typeclass/Interface] Dateable}
    If the tuple $(D,O)$ is dateable, the operator $O$, which also should be a function, 
    map on the collection $D$, and that should get the dates of $D$.
    
    For Haskell:
    \begin{haskell}
 class Dateable a where
   getDate :: a -> Date
    \end{haskell}
    and, for Java:
    \begin{java}
  interface IDateable
  {
      public abstract Date getDate();
  }
    \end{java}
    
    \subsection{[Typecalss/Interface] Itemable}
    If the tuple $(D,P_{set},P_{get},C_{set},C_{get})$ is itemable, the operators $P$, a function, map on the collection $D$,
    and that should get the priorities of $D$ or set the priorities of $D$, as well as, the operators $C$,
    a function too, map on the collection $D$, and that should get the contexts of $D$ or set the contexts of $D$.
    
    For Haskell:
    \begin{haskell}
 class Itemable a where
   getPriority :: a -> Priority
   setPriority :: a -> Priority -> a
   getContext  :: a -> Context
   setContext  :: a -> Context  -> a
    \end{haskell}
    and, for Java:
    \begin{java}
  interface IItemable
  {
      public abstract Priority getPriority();
      public abstract void     setPriority(Priority);
      public abstract Context  getContext();
      public abstract void     setContext(Context);
  }
    \end{java}
    
    \subsection{[Data] Todo}
    The \lstinline|Todo| or \lstinline|PIMTodo| is a data structure, or say class, which hold the todo within context,
    and this should be the instance of both Dateable and Itemable.
    
    For the common text interface, the followings are necessary field: 
    \begin{json}
 {
     "type":"todo",
     "date":"{DATE}",
     "context:"{STRING}"
 }
    \end{json} 
    
    \subsection{[Data] Note}
    The \lstinline|Note| or \lstinline|PIMNote| is a data structure, or say class, which hold the note within context,
    and this should be the instance of Itemable.
    
    For the common text interface, the followings are necessary field: 
    \begin{json}
 {
     "type":"note",
     "context":"{STRING}",
 }
    \end{json}
    
    \subsection{[Data] Appointment}
    The \lstinline|Appointment| or \lstinline|PIMAppointment| is a data structure, or say class,
    which hold the appointment informations within the context, and this should be instance of both Datebale and Itemable.
    
    For the common text interface, the followings are necessary field: 
    \begin{json}
 {
     "type":"appointment",
     "date":"{DATE}",
     "context":"{STRING}"
 }
    \end{json}
    
    \subsection{[Data] Contact}
    The \lstinline|Contact| or \lstinline|PIMContact| is a data structure, or say class, which hold the first name, last name,
    and email, and this should be the instance of Itemable.
    
    For the common text interface, the followings are necessary field: 
    \begin{json}
 {
     "type":"contact",
     "first-name":"{STRING}",
     "last-name":"{STRING}",
     "email":"{STRING}"
 }
    \end{json}
    
    
    \subsection{[Enum] Priority}
    The priority should have five level: \verb|right now|, \verb|urgent|, \verb|normal|, \verb|soon|, and \verb|never|.
    
    \section{Application Connection Interface}
    This section is about the standard actions for the client, which try to manage the information itself or the data itself.
    
    \subsection{Manager}
    The \lstinline|Manager| or \lstinline|PIMManager| should be a data structure, or say class, which should have the following functions.
    
    If a tuple $(list,create,save,load,D)$(or say a subtuple), whose elements are the operators and a collection of items($D$), can meet the conditions:
    \begin{itemize}
        \item For $list$, the operator $list$ can show or display the list of the elements(items) in the collection($D$).
        \item For $create$, the operator $create$ can create or add an new element(item) to the collection($D$), and get a new collection.
        \item For $save$, the operator $save$ can use \verb|remote collection| to save the datas to wherever 
        \verb|remote collection| store them.
        \item For $load$, the operator $load$ can use \verb|remote collection| to load the datas from wherever
        \verb|remote collection| store them.
    \end{itemize}
    We can just call that such tuple $manager$. 
    
    If a \textit{super-manage} $(list,create,save,load,quit,D)$ has the operator $quit$ which can quit the current state machine,
    we can say this \textit{super-manage} is regarded as \textbf{\textit{Standard Manage Form}}. 

   \subsection{Connector}
   The \lstinline|Connector|, \lstinline|PIMCollection|, or \lstinline|RemotePIMCollection| should be a data structure, or say a class,
   which should have match the followings description.

   If a tuple $(D,notes_{get},todos_{get},appointments_{get},contacts_{get})$(or say a subtuple), whose elements are a collection($D$) and the operators,
   can meet these conditions:
   \begin{itemize}
       \item For $notes_{get}$, this operator can filter the non-note items out, and get a new collection of notes.
       \item For $todos_{get}$, this operator can filter the non-todo items out, and get a new collection of todos.
       \item For $appointments_{get}$, this operator can filter the non-appointment items out, and get a new collection of appointments.
       \item For $contacts_{get}$, this operator can filter the non-contact items out, and get a new collection of appointments.
   \end{itemize}
   We can just call that such tuple $connector$, and if the subtuple of $t$ can be can a $connector$, we can just say that tuple
   is under the \textbf{\textit{Basic Connector Form(BCF for short)}}.

  If a tuple $t$ in \verb|BCF|, has an operator, named $viaData$ for example, which can filter by date and get a new collection,
  such a tuple is under the \textbf{\textit{Normal Connector Form(NCF for short)}}.

  If a tuple $t$ in \verb|BCF|, has an operator, named $all$ for example, which can just return the collection($D$) itself,
  such a tuple is under the \textbf{\textit{Self Connector Form(SCF for short)}}.

  If a tuple $t$ is under the \verb|NCF| and \verb|SCF| and has a operator, named $add$ for example, which can add a new item to the collection($D$),
  such a tuple is under the \textbf{\textit{Full Connector Form(FCF for short)}}.

  If a tuple $t$ is under the \verb|BCF|, \verb|NCF|, \verb|SCF|, or \verb|FCF|, and this tuple's filter operators can filter by owner when necessary,
  such a tuple is under a form with ownership.

  And the \lstinline|PIMCollection| and \lstinline|RemotePIMCollection| should be under \verb|FCF| with ownership.


  \section{Application Public Interface}

  \subsection{POST /add}

  \verb|POST /add| is the API to upload or change the item.

  \verb|type| is the param which point out the update item's kind.

  \verb|id|, optional, mark up the id of the item, and this means change, not the addition.

  \verb|date|, when needed, this param points out the date of the item.

  \verb|context|, when needed, this param points out the context of the item. \textit{NOTE: For contact, you should not use this.}

  \verb|priority| is the param point out priority of the item.

  \verb|email|, for contact only, this param point out the e-mail address of the item.

  \verb|last-name|, for contact only, this param point out the last name of the item.

  \verb|first-name|, for contact only, this param point out the first name of the item.

  \subsection{GET /list/todo}

  \verb|GET /list/todo| is the API to fetch the list and items of the todos' collection.

  \verb|begin| and \verb|end| are the params which limit the date of the collection of the request's result.

  \verb|own| is the param which limit the ownweship.


  \subsection{GET /list/note}

  \verb|GET /list/note| is the API to fetch the list and items of the notes' collection.

  \verb|begin| and \verb|end| are the params which limit the date of the collection of the request's result.

  \verb|own| is the param which limit the ownweship.

  \subsection{GET /list/appointment}

  \verb|GET /list/appointment| is the API to fetch the list and items of the appointments' collection.

  \verb|begin| and \verb|end| are the params which limit the date of the collection of the request's result.

  \verb|own| is the param which limit the ownweship.

  \subsection{GET /list/contact}

  \verb|GET /list/contact| is the API to fetch the list and items of the contacts' collection.

  \verb|begin| and \verb|end| are the params which limit the date of the collection of the request's result.

  \verb|own| is the param which limit the ownweship.  

  \subsection{Return}

  The return include a status code, and the list of items:
  
  \begin{json}
 {
   "status" : 0,
   "context" : [ { "type":"note","context":"abc"}]
 }
  \end{json}
  
\end{document}